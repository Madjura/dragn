% Abstract
%%=========================================
\addcontentsline{toc}{section}{Abstract}
\section*{Abstract}
Close Reading is the process of analysing a text in-depth. The goal is to get as much information as possible in general or to focus on a specific aspect or question. Franco Moretti, an Italian literary scholar, coined the term "Distant Reading". The goal of Distant Reading is to gain understanding of texts by skimming the contents to gain a more general overview of the content. Due to the nature of Distant Reading, it is possible to process more content, faster, than would be possible with Close Reading. To do so by hand without the assistance of a tool is obviously very laborious and difficult. Machine assisted work in this area can be much more efficient.\\
This thesis is built and improved upon an existing system, dubbed Skimmr, developed by Vit Novacek. The goal of the tool itself is to allow humans to effectively and efficiently perform Distant Reading on collections of texts or even just single texts. To accomplish this task, state-of-the-art technologies and frameworks are being used to improve the quality of the original system.\\
The new system, named "dragn", utilises a pipeline of four processing steps to create an information structure that users can query.
An arbitrary number of texts can be used and processed by the system. First the texts are parsed into paragraphs, Noun Phrases (NP) extracted and an inverse index of sentences is built. Using this index, a modified pointwise mutual information (PMI) value is calculated over the corpus. The modified PMI used in this system takes into account the frequency of the two tokens co-occurring and a calculated weighted distance between the two. This causes the score to be increased the more often two tokens appear close to each other in a text. Performing the score calculation this way improves the quality of results for the Distant Reading. After building a vector space over those PMI-like values, relevant tokens are computed for each of the non-stopword tokens of the texts using Cosine Similarity. Using the computed data, files are written to the disk to make it easier to perform queries on the data and to allow the data being used in different ways, such as a different front end. \\
Users can query the texts to find text passages calculated to be relevant based on the modified PMI value and Cosine Similarity relations contained therein and see how different words in the texts are connected through the same metrics. Distant Reading becomes possible by having access to those select paragraphs and being able to read previous and following passages from the text.
